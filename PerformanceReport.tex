\documentclass[12pt,letter]{article}
% poma_style.sty is used to generate the appropriate format
\usepackage{poma_style}
% Include any other packages and definitions needed
\usepackage{graphicx}
\usepackage{amsmath,amsfonts,bm}
\usepackage{courier}
\usepackage{setspace}

\begin{document}
study the trade-off between get and put latencies in a strongly consistent configuration

\onehalfspacing
\section*{}
\subsection*{i: study the trade-off between get and post latencies in a strongly consistent configuration}
For (i), you are to set N = 5 and consider all feasible (R, W ) pairs. For each quorum configuration choice, you are to report the average latency of post and get operations. Based on this, you are to come up with recommendations for the quorum parameters to choose based on the following workload features: request arrival rate, ratio of put requests, ratio of get requests.
We ran all of our testing on multiple AWS instances (accounting for the higher than expected latency). Based on our results, we recommend a read quorum of 1 and a write quorum of 4.\\

\subsection*{ii: study the trade-off between latency and freshness across configurations offering different levels of weak consistency}
We chose to analyze read-write sloppy quorums of sizes 2,2 and 1,1. We hypothesized that the fraction of the get request returning stale data would approximate the size of the write quorum compared to the replication size (in nodes). For example, with a write quorum of size 2, we would expect both read nodes to not overlap with either write node 70 percent of the time. With write and read quorums of size 1, the odds of a miss is 80 percent.\\

\onehalfspacing
We confirmed our hypothesis for the ratios of stale data. For the 1,1 case, the average post latency was 9726923 ns, and the average get latency was 5588826 ns. For the 2,2 case, the average post latency was 17302610 ns, and the average get latency was 9070155 ms. However, its worth noting that the latency calculation does not take into account the overhead associated with determining whether the response to a get was stale. Actually checking this fact messes with the result because you have to actually more nodes to determine if the received value was the newest. 

\subsection*{iii: study the efficacy of your fault tolerance mechanisms}
When the client loses connection with the server front end, it first attempts to retry by re-sending the data again and then terminates the connection entirely. If necessary, the user accessing the client is able to restart the client and it will reconnect with the server once the server front end comes back online. The timing mechanism comes into effect with regards to specific requests (get or post) and not the connection itself.\\

\onehalfspacing
When one of the back-end replicas fails for long enough that the server front end notices it's down, the server-front end terminates the connection. Once the replication node comes back on line the server front end will immediately reconnect with the newly recovered node. When the node initially fails, the front end, depending if there are "spare" replication nodes to fill out the quorum, either adds one of the spares to the quorum as necessarily (choosing at random from the spares) or continually retries (with a 10 ms delay) if there are not enough spares to fill out the quorum. There is not an explicit message sent to the client, but because it will receive the 1 and 10 second timeouts, it will know that something is causing delays in the system.
\end{document}
