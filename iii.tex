\documentclass[12pt,letter]{article}
% poma_style.sty is used to generate the appropriate format
\usepackage{poma_style}
% Include any other packages and definitions needed
\usepackage{graphicx}
\usepackage{amsmath,amsfonts,bm}
\usepackage{courier}
\usepackage{setspace}

\begin{document}
\onehalfspacing
\section*{}
\subsection*{iii: study the efficacy of your fault tolerance mechanisms}
When the client loses connection with the server front end, it first attempts to retry by re-sending the data again and then terminates the connection entirely. If necessary, the user accessing the client is able to restart the client and it will reconnect with the server once the server front end comes back online. The timing mechanism comes into effect with regards to specific requests (get or post) and not the connection itself.\\

\onehalfspacing
When one of the back-end replicas fails for long enough that the server front end notices it's down, the server-front end terminates the connection. Once the replication node comes back on line the server front end will immediately reconnect with the newly recovered node. When the node initially fails, the front end, depending if there are "spare" replication nodes to fill out the quorum, either adds one of the spares to the quorum as necessarily (choosing at random from the spares) or continually retries (with a 10 ms delay) if there are not enough spares to fill out the quorum. There is not an explicit message sent to the client, but because it will receive the 1 and 10 second timeouts, it will know that something is causing delays in the system.
\end{document}
